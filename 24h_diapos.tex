\documentclass[xcolor=table]{beamer}
\usetheme{ALUF}

\usepackage[utf8]{inputenc}
% \usepackage{palatino}
% \usepackage[T1]{fontenc}
\usepackage{lmodern}
\usepackage[protrusion=true,expansion=true,tracking=true,kerning=true]{microtype}
\usepackage{graphicx}
\usepackage[utf8]{inputenc}
\usepackage{color}
\usepackage{textcomp}
\usepackage{charter}
\usepackage[T1]{fontenc}
\usepackage[frenchb]{babel}
\usepackage{caption}
\usepackage{pdfpages}
\usepackage{array}
\usepackage{supertabular}
\usepackage{hyperref}
\usepackage{fancyhdr} 
\usepackage{eurosym}
\usepackage{multirow}
\usepackage{parallel}
\usepackage{booktabs}

\newcommand{\tabitem}{\par\hspace*{\labelsep}\textbullet\hspace*{\labelsep}}

%Permet de faire des listes dans les tableaux
\makeatletter
\newcolumntype{e}[1]{%--- Enumerated cells ---
   >{\minipage[b]{\linewidth}%
     \NoHyper%                Hyperref adds a vertical space
     \let\\\tabularnewline
     \enumerate
        \addtolength{\rightskip}{0pt plus 50pt}% for raggedright
        \setlength{\itemsep}{-\parsep}}%
   p{#1}%
   <{\@finalstrut\@arstrutbox\endenumerate
     \endNoHyper
     \endminipage}}

\newcolumntype{i}[1]{%--- Itemized cells ---
   >{\minipage[b]{\linewidth}%
        \let\\\tabularnewline
        \itemize
           \addtolength{\rightskip}{0pt plus 50pt}%
           \setlength{\itemsep}{-\parsep}}%
   p{#1}%
   <{\@finalstrut\@arstrutbox\enditemize\endminipage}}

\AtBeginDocument{%
    \@ifpackageloaded{hyperref}{}%
        {\let\NoHyper\relax\let\endNoHyper\relax}}
\makeatother

\title{24 heures de l'INSA - 44\up{ème} édition}
\subtitle{18 - 20 mai 2017}
\author{Formation Grands Rassemblements}
\date{2 mai 2018}
\institute{Arthur Saunier - 06 25 53 25 79 \\Valentin Godrie - 07 52 62 04 69\\Léo Mouyna - 06 24 30 26 53}

\begin{document}

\begin{frame}[plain,t]
\titlepage
\end{frame}

\begin{frame}% [plain,t]
	\frametitle{Ordre du jour}
\tableofcontents
\end{frame}

%=============================================================================================

%========================================
%PRESENTATION DE LA MANIFESTATION
%========================================

\section{Présentation de la manifestation}

\begin{frame}

\centering\Huge{\textbf{Présentation de la manifestation}}

\end{frame}

\begin{frame}

\frametitle{Généralités}

\setlength{\tabcolsep}{2pt}
\begin{tabular}{cccc}
\hline
\multicolumn{4}{|c|}{\centering\Large\textbf{44\up{ème} édition - 18, 19 et 20 mai 2017}}\\
\hline
\\
\includegraphics[height=.2\textheight]{Images/Image1} & \includegraphics[height=.2\textheight]{Images/Image2} & \includegraphics[height=.2\textheight]{Images/Image3} & \includegraphics[height=.2\textheight]{Images/Image4}\\
\\
\multicolumn{4}{l}{\tabitem 24 heures de courses}\\
\multicolumn{4}{l}{\tabitem 3 soirs de concerts}\\
\multicolumn{4}{l}{\tabitem 2 journées d'animations gratuites}\\
\multicolumn{4}{l}{\tabitem Environ 30 000 personnes sur 3 jours}\\

\end{tabular}

\end{frame}

\begin{frame}

\frametitle{Les 24 heures en quelques chiffres}

\begin{itemize}
\item \textbf{70} organisateurs toute l'année
\item \textbf{300} organisateurs bénévoles le week-end
\item \textbf{9000} personnes par soirée (2017)
\item \textbf{30 000} personnes attendues sur le week-end
\end{itemize}

\end{frame}

\begin{frame}

\frametitle{Plan général du site}

\centering\includegraphics[height=.8\textheight, trim=150 70 0 0,clip]{Exports/Plan_24h_43eme-Parcours_courses}

\end{frame}

\begin{frame}

\frametitle{Organisation}

\centering\includegraphics[width=\textwidth]{Images/Timeline}

\end{frame}

%========================================
%BILAN DE L'EDITION 2017
%========================================

\section{Bilan de l'édition 2017}

\begin{frame}

\centering\Huge{\textbf{Bilan de l'édition 2017}}

\end{frame}

\begin{frame}

\frametitle{Nouvelles dispositions}
\begin{itemize}
\item\textbf{Nouveau dispositif de fermeture à la circulation}

$\rightarrow$ Aucun forçage du dispositif pendant les périodes concernées (20h-5h)

\item\textbf{Ajout de silo à verre aux entrées}

$\rightarrow$ Moins de verre présent autour de cette zone

\item\textbf{Augmentation et répartition des effectifs d'AS en journée}

$\rightarrow$ Aucun incident notable pendant les deux journées d'animations
\end{itemize}
\end{frame}

\begin{frame}

\frametitle{Remarques générales}
\begin{itemize}
\item Public majoritairement étudiant en soirée
\item Très peu de problèmes rencontrés sur l'ensemble du week-end
\item Plusieurs pickpockets arrêtés par la Police, une 50aine de téléphones récupérés
\item Quelques points de passage forcés par des véhicules en journée
\item Quelques collisions cyclistes/spectateurs
\end{itemize}
\end{frame}

\begin{frame}

\frametitle{Affluence}
\rowcolors[\hline]{2}{blue!30}{blue!10}
\begin{tabular}{|l|c|c|c|c|}
\hline
\rowcolor{blue!50} &\bf2015&\bf2016&\bf2017&\bf\parbox[c]{1.7cm}{\centering{Variation}\\2016/2017}\\
\hline
Vendredi soir  & 8700 & 8400 & 7600 & -10\%\\
\hline
Samedi soir  & 8000 & 9650 & 8400 & -13\%\\
\hline
Dimanche soir & 900 & 1600 & 1000 & -60\%\\
\hline
Total soirées & 17 600 & 19 650 & 17 000 & -13\%\\
\hline
Samedi journée & 5000 & 4000 & 5000 & +25\%\\
\hline
Dimanche journée & 5000 & 5000 & 5000 & =\\
\hline
Total journées  & 10 000 & 9000 & 10 000 & +11\%\\
\hline
Total festival  & 27 600 & 28 650 & 27 000 & -6\%\\
\hline

\end{tabular}

\end{frame}

\begin{frame}

\frametitle{Interventions}

\rowcolors[\hline]{2}{blue!30}{blue!10}
\begin{tabular}{|l|c|c|c|c|}
\hline
\rowcolor{blue!50} &\bf2015&\bf2016&\bf2017&\bf\parbox[c]{1.7cm}{\centering{Variation}\\2016/2017}\\
\hline
Croix-Rouge & 101 & 68 & 69 & =\\
\hline
STAFF Sécurité & 14 & 18 & 5 & -72\%\\
\hline
\end{tabular}

\end{frame}

\begin{frame}

\frametitle{Nouveautés 2018}
\framesubtitle{Prise en compte du bilan 2017}

\end{frame}

%========================================
%DISPOSITIF DE SECURITE
%========================================

\section{Dispositif de sécurité}

\begin{frame}

\centering\Huge{\textbf{Dispositif de sécurité}}

\end{frame}

\begin{frame}

\frametitle{Plans de situation}
\begin{Parallel}{.5\textwidth}{.5\textwidth}
\ParallelLText{\centering\includegraphics[height=.7\textheight, trim=300 20 0 0,clip]{Exports/Plan_24h_43eme-Plan_ERP}}
\ParallelRText{\centering\includegraphics[height=.7\textheight, trim=0 0 0 0,clip]{Exports/ERP_vue_aerienne}}
\end{Parallel}

\end{frame}

\begin{frame}

\frametitle{Zone ERP}

\begin{tabular}{i{.5\textwidth} c}
\item Validation de la zone à 30 900 personnes le par la sous-commission ERP-IGH le 12 avril 2017
\vspace{0.5cm}
\item Exploitation à 10 000 personnes
\vspace{1cm}
& \includegraphics[height=.8\textheight, trim=0 0 0 0,clip]{Exports/Plan_24h_43eme-Plan_de_situation}\\
\end{tabular}

\end{frame}

\begin{frame}

\frametitle{Dégagements}
\begin{tabular}{i{.3\textwidth} c}
\item \textbf{13 sorties} pour un minimum de 13 imposées
\vspace{0.5cm}
\item \textbf{113 UP} pour un minimum de 103 imposées
\vspace{2cm}
& \includegraphics[width=.65\textwidth, trim=100 50 150 0,clip]{Exports/Plan_24h_43eme-IS}\\
\end{tabular}

\end{frame}

\begin{frame}

\frametitle{Voies pompiers}
\begin{tabular}{i{.35\textwidth} c}
\item \textbf{6 voies pompiers} autour de la zone ERP
\vspace{0.45cm}
\item \textbf{Agents de sûreté} préposés à l'ouverture en liaison avec le PC Sécurité
\vspace{0.45cm}
\item Procédures d'ouverture des accès rapide et fiables
& \includegraphics[width=.6\textwidth, trim=130 0 180 0,clip]{Exports/Plan_24h_43eme-Acces_Pompiers}\\
\end{tabular}

\end{frame}

\begin{frame}

\frametitle{Protection contre l'incendie}
\begin{tabular}{i{.4\textwidth} c}
\item \textbf{Agents de sécurité} incendie : 1 SSIAP 3, 2 SSIAP 2, 4 SSIAP 1
\vspace{0.3cm}
\item \textbf{28 exctinteurs} à disposition
\vspace{0.3cm}
\item \textbf{Formation} de 20 bénévoles à leur utilisation
\vspace{0.3cm}
\item Diffusion d'un message sonore et visuel en cas d'évacuation de la zone, \textbf{arrêt des activités}
& \includegraphics[width=.55\textwidth, trim=150 0 190 0,clip]{Exports/Plan_24h_43eme-Extincteurs}\\
\end{tabular}

\end{frame}

\begin{frame}

\frametitle{Dispositions spécifiques à l'espace restauration}
\begin{tabular}{i{.50\textwidth} c}
\item \textbf{5-6} concessions salées
\item \textbf{2 concessions sucrées}
\item Ouverture en \textbf{soirée de 20h à 3h45} et en \textbf{journée de 10h à 18h}
\item Foodtruck et tentes \textbf{ignifugées}
\item Présence d'agents de sûreté pour réguler le flux de personnes
\item Extincteurs A3F, H20 et CO2 à disposition sur place
& \includegraphics[width=.4\textwidth, trim=0 0 0 0,clip]{Exports/Plan_24h_43eme-Espace_Resto_zoom}\\
\end{tabular}

\end{frame}

\begin{frame}

\frametitle{Troisième scène - Implantation}
\begin{tabular}{i{.35\textwidth} c}
\item \textbf{Principe :} Petite scène, public de 500 personnes environ
\vspace{0.45cm}
\item En soirée : \textbf{20h-3h20}
\vspace{0.45cm}
\item En journée (samedi et dimanche) : \textbf{14h-17h}
\vspace{.5cm}
& \includegraphics[width=.6\textwidth, trim=300 100 100 0,clip]{Exports/Plan_24h_43eme-3e_Scene}\\
\end{tabular}

\end{frame}

\begin{frame}

\frametitle{Troisième scène - Dispositions de sécurité}
\begin{tabular}{i{.35\textwidth} c}
\item \textbf{2 sorties} de 4m
\vspace{0.45cm}
\item Mise en place de \textbf{moyens d'extinction} appropriés (CO2 + eau)
\vspace{0.45cm}
\item Présence de \textbf{2 agents de sûreté} pendant l'animation
& \includegraphics[width=.6\textwidth, trim=0 0 0 0,clip]{Exports/Plan_24h_43eme-3e_Scene_Secu_Incendie}\\
\end{tabular}

\end{frame}

\begin{frame}

\frametitle{Emplacements PCs \& Croix-Rouge Française}
\begin{tabular}{i{.4\textwidth} c}
\item \textbf{Liaison permanente} avec l'équipe sécurité
\item \textbf{Secouristes} présents pour les soirées de concerts
\item \textbf{Equipes mobiles} patrouillant sur la zone concerts
\item \textbf{Poste de secours} aménagé dans le Gymnase C
\item \textbf {Redirection} des appels du \textbf{CTA}
\item Liaison avec le \textbf{SAMU (15)}
& \includegraphics[width=.55\textwidth, trim=350 0 50 0,clip]{Exports/Plan_24h_43eme-PC_Secu}\\
\end{tabular}

\end{frame}

\begin{frame}

\frametitle{Installations électriques}
\begin{itemize}
\item {Eclairage des sorties :}
\begin{itemize}
\item \textbf{Les sorties sont éclairées en permanence}
\item En cas de coupure de courant ou en cas d'évacuation, des \textbf{blocs phares} branchés sur batterie \textbf{éclairents les sorties de secours}
\end{itemize}
\item {Installations électriques :}
\begin{itemize}
\item PRésence d'un \textbf{électricien professionnel}
\item Vérifiées par un \textbf{organisme de contrôle} (OCdS)
\item \textbf{Aucune installation n'est à portée du public}
\end{itemize}
\end{itemize}

\end{frame}

\begin{frame}

\frametitle{Zone de caisses}
\begin{tabular}{e{.4\textwidth} c}
\item \textbf{Contrôle visuel} des billets et des sacs par des agents de sûreté
\vspace{.5cm}
\item \textbf{Palpation} par des agents de sûreté
\vspace{.5cm}
\item \textbf{Contrôle} des billets
\vspace{.5cm}
\item \textbf{Accès Zone ERP}
& \includegraphics[width=.55\textwidth, trim=0 0 0 0,clip]{Exports/Plan_24h_43eme-Entree_zoom}\\
\end{tabular}

\end{frame}

\begin{frame}

\frametitle{Grande scène}
\begin{itemize}
\item Etude et montage de la scène par la société Scénétec
\item Vérification par un \textbf{organisme de contrôle} (SOCOTEC)
\end{itemize}
\includegraphics[width=\textwidth]{Images/scene}

\end{frame}

\begin{frame}

\frametitle{Risques météorologiques : Pluie}
\textbf{Ne crée pas de risques supplémentaires car :}
\begin{itemize}
\item Installations électriques dans des coffrets \textbf{étanches}
\item Matériel sensible sous \textbf{tentes}
\end{itemize}
\end{frame}

\begin{frame}

\frametitle{Risques météorologiques : Vent}
\begin{itemize}
\item \textbf{Surveillance} : anémomètre, veille météo
\item Vent \textbf{faible} :
\begin{itemize}
\item Tout est \textbf{accroché} (signalétique, banderoles…)
\item Bars \textbf{lestés} (500kg par bar) 
\item Clôtures fixées (jambe de force ou poteau)
\item Bâches micro-perforées
\end{itemize}
\item Vent fort :
\begin{itemize}
\item Scène validée jusqu’à 72 km/h en vent établi
\item Descente du toit au-delà 
$\rightarrow$ Annulation des concerts et évacuation de la zone
\end{itemize}
\end{itemize}

\end{frame}

%========================================
%PROCEDURES D'URGENCE
%========================================

\section{Procédures d'urgence}

\begin{frame}

\centering\Huge{\textbf{Procédures d'urgence}}

\end{frame}

\begin{frame}

\frametitle{Plan d'action en cas d'urgence}
\includegraphics[width=\textwidth]{Images/procedures}

\end{frame}

\begin{frame}

\frametitle{Plan d'accès des secours}
\begin{tabular}{i{.35\textwidth} c}
\item \textbf{Adresse unique :} 20 avenue Albert Einstein
\item \textbf{Point de ralliement :} poste de garde de l'INSA
\item \textbf{PMA possibles :} Gymnases, hall Marie Curie, bâtiment le Galilée
& \includegraphics[width=.6\textwidth, trim=0 0 150 0,clip]{Exports/Plan_24h_43eme-PCO_PMA}\\
\end{tabular}

\end{frame}

\begin{frame}

\frametitle{Evacuation des PMR}
\begin{itemize}
\item Présence d’un agent \textbf{SSIAP 1} à proximité de la rampe PMR et présence d’une dizaine d’organisateurs à la buvette à proximité.
\vspace{.5cm}
\item Message d’évacuation \textbf{sonore et visuel} pour les malentendants et malvoyants
\vspace{.5cm}
\item Agents de sûreté aux issues de secours
\end{itemize}

\end{frame}

%========================================
%DISPOSITIF DE SURETE
%========================================

\section{Dispositif de sûreté}

\begin{frame}

\centering\Huge{\textbf{Dispositif de sûreté}}

\end{frame}

\begin{frame}

\frametitle{L'équipe sécurité des 24 heures de l'INSA}
\begin{itemize}
\item Mise en place d'un \textbf{PC sécurité}
\vspace{.3cm}
\item \textbf{Permanence 24h/24}
\vspace{.3cm}
\item Tenue de la \textbf{main courante}
\vspace{.3cm}
\item \textbf{Coordination} des différents acteurs :
\begin{itemize}
\item Présence des responsables de \textbf{STAFF Sécurité} et \textbf{Croix-Rouge} 
\item \textbf{Liaison radio} avec les organisateurs
\item Liaison avec les \textbf{services de secours}
\item Liaison avec le service de prevention \& de sécurité de l’INSA
\item Liaison avec le PC sécurité TCL
\end{itemize}
\end{itemize}

\end{frame}

\begin{frame}

\frametitle{Les agents de sûreté (AS)}
\begin{tabular}{i{.4\textwidth} c}
\item \textbf{Société STAFF Sécurité}
\item \textbf{Agents de sécurité} incendie : 1 SSIAP 3, 2 SSIAP 2, 4 SSIAP 1
\item Jusqu'à \textbf{92 AS} aux heures de pointe
\item \textbf{Liaison radio permanente} avec l'équipe sécurité
\item \textbf{6 équipes cynophiles} (1 maître-chien + 1 AS + 1 chien)
& \includegraphics[width=.55\textwidth, trim=350 0 30 50,clip]{Exports/Plan_24h_43eme-AS_Nuit}\\
\end{tabular}


\end{frame}

\begin{frame}

\frametitle{Les agents de sûreté (AS) - Zone des caisses}
\begin{tabular}{i{.45\textwidth} c}
\item \textbf{Equipe mobile} : 1 SSIAP 1 + 1 agent + 1 maître-chien
\item \textbf{Palpation} 12 agents
\item \textbf{Files} 6 agents
\item \textbf{Sortie} 4 agents
\item \textbf{Caisses} 3 agents
\item \textbf{Alentours} : 2 maîtres-chiens + 1 AS
\item \textbf{Total} : 31 agents sur la zone
\item \textbf{Evacuation de l'argent :} Couloir de circulation en zone sécurisée
& \includegraphics[width=.5\textwidth, trim=100 0 50 0,clip]{Exports/Plan_24h_43eme-Entree_AS}\\
\end{tabular}

\end{frame}

\begin{frame}

\frametitle{Dispositions spécifiques liées à la "Travée Verte"}
\begin{tabular}{i{.4\textwidth} c}
\item Dispositif en place depuis 4 ans avec d'excellents résultats
\vspace{.3cm}
\item Mis en place les \textbf{vendredi et samedi de 20h à 9h}
\vspace{.3cm}
\item Présence d'agents pour ouvrir les \textbf{voies pompiers}
& \includegraphics[width=.55\textwidth, trim=0 80 350 100,clip]{Exports/Plan_24h_43eme-AS_Nuit}\\
\end{tabular}

\end{frame}

\begin{frame}

\frametitle{Prise en compte du plan Vigipirate}

\end{frame}

%========================================
%ACCESSIBILITE
%========================================

\section{Accessibilité}

\begin{frame}

\centering\Huge{\textbf{Accessibilité}}

\end{frame}

\begin{frame}

\frametitle{Dispositif d'accessibilité}
\begin{tabular}{i{.5\textwidth} c}
\item Dispositif validé en \textbf{SCDA} en 2014
\vspace{1mm}
\item Parking réservé aux PMR à proximité des entrées
\vspace{1mm}
\item Rampe surélevée avec présence de SSIAP 1
\vspace{1mm}
\item Toilette PMR situées au niveau des toilettes du public Nord
\vspace{1mm}
\item Caisse et buvette aménagées pour les PMR
\vspace{1mm}
\item Signalétique adaptée
& \includegraphics[width=.45\textwidth, trim=0 0 0 30,clip]{Exports/Plan_24h_43eme-PMR_sud}\\
\end{tabular}

\end{frame}

\begin{frame}

\frametitle{Dispositif d'accessibilité}
\begin{tabular}{i{.5\textwidth} c}
\item Dispositif validé en \textbf{SCDA} en 2014
\vspace{1mm}
\item Parking réservé aux PMR à proximité des entrées
\vspace{1mm}
\item Rampe surélevée avec présence de SSIAP 1
\vspace{1mm}
\item Toilette PMR situées au niveau des toilettes du public Nord
\vspace{1mm}
\item Caisse et buvette aménagées pour les PMR
\vspace{1mm}
\item Signalétique adaptée
& \includegraphics[width=.45\textwidth, trim=0 0 0 30,clip]{Exports/Plan_24h_43eme-PMR_nord}\\
\end{tabular}

\end{frame}

%========================================
%PREVENTION SUR LE CAMPUS
%========================================

\section{Prévention sur le campus}

\begin{frame}

\centering\Huge{\textbf{Prévention sur le campus}}

\end{frame}

\begin{frame}

\frametitle{Dispositions concernant le campus}
\begin{itemize}
\item Services médicaux et de sécurité civile à proximité :
\begin{itemize}
\item \textbf{Services de police autorisés à intervenir} sur le campus
\item \textbf{Pompiers} informés
\item Proximité de la \textbf{clinique du Tonkin} 
\end{itemize}
\item Bâtiments et laboratoires : 
\begin{itemize}
\item \textbf{Renforcement} du Service de Prévention et de sécurité de l’INSA (4 agents supplémentaires)
\item \textbf{Renforcement} du Service de Prévention et de sécurité de l’UCBL (2 agents supplémentaires)
\item Astreinte de la Direction du Patrimoine de l’INSA \textbf{doublée} (technique et électrique)
\end{itemize}
\end{itemize}

\end{frame}

\begin{frame}

\frametitle{Transports en commun}
\begin{itemize}
\item Mesures de prévention :
\begin{itemize}
\item Blocage des emplacements dangereux
\item Agents de sûreté en fin de soirée aux abords des stations de tramway
\end{itemize}
\item Contact avec le PC sécurité TCL:
\begin{itemize}
\item En cas d’évacuation 
\end{itemize}
\end{itemize}

\end{frame}

\begin{frame}

\frametitle{Protection des bâtiments sur le campus}
\begin{tabular}{i{.35\textwidth} c}
\item A compléter
& \includegraphics[width=.6\textwidth, trim=0 0 0 30,clip]{Exports/Plan_24h_43eme-Zone_de_travaux}\\
\end{tabular}

\end{frame}


\begin{frame}

\frametitle{Sécurisation des zones de travaux}
\includegraphics[width=.9\textwidth, trim=0 0 0 0,clip]{Exports/Plan_24h_43eme-Protection_Bat}

\end{frame}

\begin{frame}

\frametitle{Avant la manifestation}
\begin{itemize}
\item Rappel des précautions à prendre en raison de l’affluence exceptionnelle :
\begin{itemize}
\item Bien fermer les logements et locaux
\item Faire attention aux vols à la tire	
\end{itemize}
\item Pose d’affiches de sensibilisation sur les supports d’affichage du campus et dans les résidences. 
\item Communication par voie électronique à l’ensemble des étudiants et personnels (détaillée plus loin)
\item Désactivation des badges d’accès aux bâtiments (de cours \& laboratoires) et des ascenseurs 
\end{itemize}

\end{frame}

\begin{frame}

\frametitle{Pendant la manifestation}
\begin{itemize}
\item Présence d’organismes de prévention :
\begin{itemize}
\item Avenir Santé – Génération Cobayes
\end{itemize}
\item Thèmes abordés :
\begin{itemize}
\item Alcool, drogues
\item Sécurité routière (opération SAM)
\item Sexualité, IST
\item Comportements à risques
\end{itemize}
\item Equipes mobiles pendant les concerts
\item Diffusion de messages de prévention sur les écrans de la scène
\end{itemize}

\end{frame}

\begin{frame}

\frametitle{Emplacement du stand de prévention}
Intégrer ici le plan d'emplacement du stand prévention
%\includegraphics[width=.9\textwidth, trim=0 0 0 0,clip]{Exports/

\end{frame}

\begin{frame}

\frametitle{Dispositions spécifiques au CETIAT}
\begin{tabular}{i{.35\textwidth} c}
\item Centre technique situé sur l'avenue des Arts
\vspace{1mm}
\item Stockage de divers matériaux \textbf{inflammables}
\vspace{1mm}
\item Gardiennage le vendredi, samedi de 20h à 4h et dimanche soir de 20h à 23h30
\vspace{1mm}
\item Liaison radio permanente avec le PC sécurité
& \includegraphics[width=.6\textwidth, trim=0 0 0 30,clip]{Exports/Plan_24h_43eme-CETIAT}\\
\end{tabular}

\end{frame}

%========================================
%COURSES ET ANIMATIONS
%========================================

\section{Courses et animations}

\begin{frame}

\centering\Huge{\textbf{Courses et animations}}

\end{frame}

\begin{frame}

\frametitle{Course cylcliste et à pied}
\begin{tabular}{i{.5\textwidth} i{.5\textwidth}}
\item \textbf{3 points} de passage
\item \textbf{2 organisateurs} avec chasuble par point de passage
& \item Rondes régulière en \textbf{voiture-balai}
\item Véhicule de la \textbf{Croix-Rouge} présent avec \textbf{3 secouristes} pendant la durée de la course
\end{tabular}
\includegraphics[width=\textwidth, trim=0 300 0 0,clip]{Exports/Plan_24h_43eme-Parcours_courses}

\end{frame}

\begin{frame}

\frametitle{Course cycliste et à pied}
\begin{itemize}
\item \textbf{Risques :}
\begin{itemize}
\item Chutes (fatigue, vitesse)
\item Collisions avec les spectateurs
\end{itemize}
\item \textbf{Mesures préventives :}
\begin{itemize}
\item \textbf{Port du casque} et éclairage de vélo obligatoires 
\item \textbf{Vérification} des certificats médicaux / Licences Sportives
\item \textbf{Barrièrage} et signalisation des points sensibles
\item \textbf{Points de passage surveillés} pour les spectateurs
\item Équipe de \textbf{secouristes} présente sur le parcours
\item \textbf{Surveillance} constante par une voiture balai
\item \textbf{Éclairage} de la course pendant la nuit
\end{itemize}
\end{itemize}
\end{frame}

\begin{frame}

\frametitle{Animations}
\begin{itemize}
\item Environ \textbf{60 animations} regroupées en 3 pôles
\item Horaires : \textbf{10h-18h} les 20 et 21 mai
\item \textbf{2 responsables animations} en liaison radio avec le PC sécurité
\end{itemize}
\begin{tabular}{i{.35\textwidth} c}
\item \textbf{Patrouilles} d'agents de sûreté
\item Agents de sûreté \textbf{fixes} aux points sensibles
\vspace{1.5cm}
& \includegraphics[width=.6\textwidth, trim=0 0 0 0,clip]{Images/animations}\\
\end{tabular}


\end{frame}

\begin{frame}

\frametitle{Répartition des AS en journée}
\includegraphics[width=\textwidth, trim=0 0 0 30,clip]{Exports/Plan_24h_43eme-AS_Jour}

\end{frame}

\begin{frame}

\frametitle{Animations sensibles}
\begin{itemize}
\item Feu d’artifice  / Spectacle de feu
\item Animation Phare
\item Dispositif de protection spécifique :
\begin{itemize}
\item Barrièrage de la zone 
\item Agents de sûreté dédiés
\item Extincteurs
\end{itemize}
\end{itemize}

\end{frame}

\begin{frame}

\frametitle{Animations sensibles - Feu d'artifice}
\begin{itemize}
\item \textbf{Spectacle de feu}
\begin{itemize}
\item Même emplacement que le feu d’artifice 
\item 10 mètres de séparation avec le public
\item Spectacle le 21 mai à 22h30
\end{itemize}
\item \textbf{Feu d’artifice}
\begin{itemize}
\item Type K4
\item 25,08 kg de matière active 
\item 95 mètres de séparation avec le public
\item Tir le \textbf{20 mai à 23h00}
\item Société ShowTechnic, comme les années précédentes
\item Public : environ 500 personnes
\end{itemize}
\end{itemize}

\end{frame}

\begin{frame}

\frametitle{Grand 8}

\end{frame}

\begin{frame}

\frametitle{Fatal Balayette}

\end{frame}

\begin{frame}

\frametitle{Soirée du dimanche}
\begin{tabular}{i{.35\textwidth} c}
\item Dimanche 20 mai de \textbf{20h à 22h30}
\vspace{1mm}
\item Concerts de \textbf{jazz}
\vspace{1mm}
\item Communication \textbf{réduite}
\vspace{1mm}
\item Mise en place d'une \textbf{zone ERP réduite}
& \includegraphics[width=.6\textwidth, trim=200 0 0 30,clip]{Exports/Plan_24h_43eme-ERP_Dimanche}\\
\end{tabular}

\end{frame}

\begin{frame}

\frametitle{Soirée du dimanche}

\end{frame}

%========================================
%CIRCULATION SUR LE CAMPUS
%========================================

\section{Circulation sur le campus}

\begin{frame}

\centering\Huge{\textbf{Circulation sur le campus}}

\end{frame}

\begin{frame}

\frametitle{}

\end{frame}


\end{document}
